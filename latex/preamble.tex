%%%%%%%%%%%%%%%  Start of preamble.tex %%%%%%%%%%%%%%%%%% 
%%%%%%%%%%%%%%%  --------------------- %%%%%%%%%%%%%%%%%%
\usepackage{booktabs}
\usepackage{longtable}
\usepackage[bf,singlelinecheck=off]{caption}
\usepackage{url}

\usepackage{framed}
\usepackage{color}  % is this needed?
\definecolor{shadecolor}{RGB}{248,248,248}
\usepackage[cmyk]{xcolor} % CRC wants CMYK
% custom colors
\definecolor{darkgreen}{RGB}{1,50,32}

%%%% section headings %%%%


% https://tex.stackexchange.com/questions/750528/how-to-change-color-of-section-title-in-the-krantz-cls/750529#750529
\renewcommand\SectionHeadFont{\color{blue}\fontsize{12}{14}\sffamily\bfseries\selectfont}
\renewcommand\SubsectionHeadFont{\color{blue}\fontsize{11}{13}\sffamily\bfseries\selectfont}
\renewcommand\SubsubsectionHeadFont{\color{blue}\fontsize{10}{12}\sffamily\bfseries\selectfont}


% Define a custom color
% dark greenish-brown color, with a hexadecimal equivalent of #1A7F33
\definecolor{mygreen}{rgb}{0.1, 0.5, 0.2} 

%%%% Page headers %%%%%%
% override krantz default to use chapter/section headers
\makeatletter
\def\HeadingsChapterSection{%
  \def\chaptermark##1{%
    \markboth{%
      \thechapter. ##1}{}}%
  \def\sectionmark##1{%
    \markright{%
      \thesection: ##1}}}
\HeadingsChapterSection


% block quotes
\definecolor{quotecolor}{RGB}{0,0,100} % Example: blue
\renewenvironment{quote}
{%
  \list{}{\rightmargin=\leftmargin}%
  \item\relax\color{quotecolor}% <-- Apply the color here
  \sffamily
}
{%
  \endlist
}

%--- Cover page
% Run:
%   pdftk A=images/cover-ellipse.pdf B=index.pdf cat A1 B2-end output index-cover.pdf
%   BUT: this changes the structure of the PDF-- no bookmarks, etc.,
%   What is needed is to open index.pdf and INSERT the cover as the first page.
%   pdftk A=index.pdf B=images/cover-ellipse.pdf cat A0 B1 A2-end output index-cover.pdf
%   --> Did this manually in Adobe Acrobat Pro WORKED

% floats
\renewcommand{\textfraction}{0.05}
\renewcommand{\topfraction}{0.8}
\renewcommand{\bottomfraction}{0.8}
\renewcommand{\floatpagefraction}{0.75}

% for tinytable / modelsummary
\usepackage{tabularray}
\usepackage{float}
\usepackage{graphicx}
\usepackage{rotating}
\usepackage[normalem]{ulem}
\UseTblrLibrary{booktabs}
\UseTblrLibrary{siunitx}
% \newcommand{\tinytableTabularrayUnderline}[1]{\underline{#1}}
% \newcommand{\tinytableTabularrayStrikeout}[1]{\sout{#1}}
% \NewTableCommand{\tinytableDefineColor}[3]{\definecolor{#1}{#2}{#3}}


% Make \href{} into a footnote --- does this work??
\let\oldhref\href
\providecommand{\href}[2]{#2\footnote{\url{#1}}}
  
% is this needed???
% \makeatletter
% \newenvironment{kframe}{%
%   \medskip{}
%   \setlength{\fboxsep}{.8em}
%   \def\at@end@of@kframe{}%
%   \ifinner\ifhmode%
%   \def\at@end@of@kframe{\end{minipage}}%
%   \begin{minipage}{\columnwidth}%
%   \fi\fi%
%   \def\FrameCommand##1{\hskip\@totalleftmargin \hskip-\fboxsep
%   \colorbox{shadecolor}{##1}\hskip-\fboxsep
%     % There is no \\@totalrightmargin, so:
%       \hskip-\linewidth \hskip-\@totalleftmargin \hskip\columnwidth}%
%   \MakeFramed {\advance\hsize-\width
%     \@totalleftmargin\z@ \linewidth\hsize
%     \@setminipage}}%
% {\par\unskip\endMakeFramed%
%   \at@end@of@kframe}
% \makeatother
% 
% \renewenvironment{Shaded}{\begin{kframe}}{\end{kframe}}

% Suppress underfull \hbox \vbox
\setlength{\hbadness}{1000pt}
\setlength{\vbadness}{1000pt}

% Define needed colors for xcolor
\definecolor{darkgreen}{RGB}{1,50,32}

% shut up overfull boxes

\hfuzz=5.002pt

% \tableofcontents becomes undefined. Use def from article.cls
% NO LONGER NEED THIS
% \makeatletter
% \renewcommand{\contentsname}{Contents}% For example
% \providecommand{\tableofcontents}{%
%   \section*{\contentsname}
%   \@starttoc{toc}
% }


%%%%%%%%%%%%%%%%%%%%%%%%%%%%%%%%%%%%%%%%%%%%%%%%%%%%%%%%%%%%%%%%%%%%%%%
% Index generation
% Indexentry for a word/phrase (Word inserted into the text)
%%%%%%%%%%%%%%%%%%%%%%%%%%%%%%%%%%%%%%%%%%%%%%%%%%%%%%%%%%%%%%%%%%%%%%%

% \usepackage{makeidx}
% \makeindex

\usepackage[small,firstabbrev]{authorindex}
\aimaxauthors{5}   % maximum number of authors to index

\newcommand{\IX}[1]{\index{#1}#1}
\newcommand{\ix}[1]{\index{#1}}
\newcommand{\ixmain}[1]{\index{#1|textbf}}
\newcommand{\ixon}[1]{\index{#1|(}}
\newcommand{\ixoff}[1]{\index{#1|)}}

% R functions
\newcommand{\ixfunc}[1]{%
  \index{#1@\texttt{#1()}}%
 }

% R packages:  indexed under both package name and packages!
\newcommand{\ixp}[1]{%
   \index{#1@\texttt{#1} package}%
   \index{packages!#1@\texttt{#1}}%
	}

% data sets: 
\newcommand{\ixd}[1]{%
  \index{#1}
  \index{datasets!#1}
  % \index{datasets!@\texttt{#1}}
  % \index{@\texttt{#1} dataset}
  }


\urlstyle{tt}

%%%%%%%%%%%%%%%%%
% Subject index
\usepackage{imakeidx}
\makeindex[title=Subject Index,columns=2,intoc=true,options=-s latex/book.ist]

% Author index
% Author index: now using the authorindex package
\usepackage[small,firstabbrev]{authorindex}
\aimaxauthors{5}   % maximum number of authors to index
% to generate the book.ain file, run:
% perl authorindex -d book
%    requires: BIBINPUTS, BSTINPUTS
%    setx BSTINPUTS "C:/Dropbox/localtexmf/bibtex/bst"
%    setx BIBINPUTS "C:/R/Projects/Vis-MLM-book/bib"


\usepackage{amsthm}
\makeatletter
\def\thm@space@setup{%
  \thm@preskip=8pt plus 2pt minus 4pt
  \thm@postskip=\thm@preskip
}

% Fix for modelsummary problem with \num{}
\renewcommand*{\num}[1]{#1}


\makeatother

% This doesn't work
% l.300 \includegraphics
%                       [width=\textwidth]{images/cover-ellipse.jpg}

%\includegraphics[width=\textwidth]{images/cover-ellipse.jpg}

\frontmatter
  
%%%%%%%%%%%%%%%  End of preamble.tex %%%%%%%%%%%%%%%%%%%%  
%%%%%%%%%%%%%%%  ------------------- %%%%%%%%%%%%%%%%%%%%
