\documentclass[11pt]{article}
%\usepackage{sectsty}
\usepackage{graphicx}
\usepackage{amssymb,amsmath}
\usepackage{xspace}
\title{Using \LaTeX\ Shorthands for Math Notation}
\author{Michael Friendly}
\date{\today}

\begin{document}
\maketitle

% math stuff
\renewcommand*{\vec}[1]{\ensuremath{\mathbf{#1}}}
%\newcommand{\trans}{\ensuremath{^\mathsf{T}}}
\newcommand{\trans}{^\top}
\newcommand*{\mat}[1]{\ensuremath{\mathbf{#1}}}
\newcommand*{\diag}[1]{\ensuremath{\mathrm{diag}\, #1}}
%\renewcommand*{\det}[1]{\ensuremath{\mathrm{det} (#1)}}
%\renewcommand*{\det}[1]{\ensuremath{|#1|}}
\renewcommand*{\det}[1]{\mathrm{det}(#1)}
\newcommand*{\detbracket}[1]{\mathrm{det} [#1]}
\newcommand*{\rank}[1]{\ensuremath{\mathrm{rank} (\mathbf{#1})}}
\newcommand*{\trace}[1]{\ensuremath{\mathrm{tr} (\mathbf{#1})}}
\newcommand*{\dev}[1]{(#1 - \bar{#1})}
\newcommand*{\inv}[1]{\ensuremath{\mat{#1}^{-1}}}
\newcommand*{\half}[1]{\ensuremath{\mat{#1}^{1/2}}}
\newcommand*{\invhalf}[1]{\ensuremath{\mat{#1}^{-1/2}}}
\newcommand*{\nvec}[2]{\ensuremath{{#1}_{1}, {#1}_{2},\ldots,{#1}_{#2}}}
\newcommand*{\Beta}{B}
\newcommand*{\Epsilon}{E}
\newcommand*{\period}{\:\: .}
\newcommand*{\comma}{\:\: ,}
\newcommand*{\given}{\ensuremath{\, | \,}}
\newcommand*{\Real}[1]{\mathbb{R}^{#1}}
\newcommand*{\degree}[1]{\ensuremath{{#1}^{\circ}}}
\newcommand{\sizedmat}[2]{%
  \mathord{\mathop{\mat{#1}}\limits_{(#2)}}%
}
\renewcommand*{\H}{\ensuremath{\mathbf{H}}\xspace}      % what am I overriding here?
\newcommand*{\E}{\ensuremath{\mathbf{E}}\xspace}
\newcommand*{\widebar}[1]{\overline{#1}}

\newcommand{\Var}{\ensuremath{\mathsf{Var}}}
\newcommand{\Cov}{\ensuremath{\mathsf{Cov}}}
\newcommand{\argmin}{\mathop{\mathrm{argmin}}}
\newcommand{\dfn}[1]{\emph{#1}}

\newcommand{\HO}{\ensuremath{\mathcal{H_0}}}

%%%%%%%%%%%%%%%%%%%%%%%%%%%%%%%%%%%%%%%%%%%%%%%%%%%%%%%%%%%%%%%%%%%%%%%
% Index generation
% Indexentry for a word/phrase (Word inserted into the text)
%%%%%%%%%%%%%%%%%%%%%%%%%%%%%%%%%%%%%%%%%%%%%%%%%%%%%%%%%%%%%%%%%%%%%%%
\newcommand{\IX}[1]{\index{#1}#1}
\newcommand{\ix}[1]{\index{#1}}
\newcommand{\ixmain}[1]{\index{#1|textbf}}

\newcommand{\ixon}[1]{\index{#1|(}}
\newcommand{\ixoff}[1]{\index{#1|)}}

% R packages:  indexed under both package name and packages!
\newcommand{\ixp}[1]{%
   \index{#1@\textsf{#1} package}%
   \index{package!#1@\textsf{#1}}%
	}

% data sets: 
\newcommand{\ixd}[1]{%
        \index{data sets!#1}}


% R stuff
\newcommand{\pkg}[1]{\textsf{#1}}
\newcommand{\Rpackage}[1]{\pkg{#1} package}
\let\proglang=\textsf
\newcommand{\R}{\textsf{R}\xspace}





In \LaTeX\ documents, I use a standard set of shorthands to keep math notation \emph{consistent} and allow me to easily change how something is represented throughout an entire document.

\begin{verbatim}
\renewcommand*{\vec}[1]{\ensuremath{\mathbf{#1}}}
\newcommand{\trans}{\ensuremath{^\mathsf{T}}}
\newcommand*{\mat}[1]{\ensuremath{\mathbf{#1}}}
\newcommand*{\diag}[1]{\ensuremath{\mathrm{diag}\, #1}}
\renewcommand*{\det}[1]{\mathrm{det}(#1)}
\newcommand*{\rank}[1]{\ensuremath{\mathrm{rank} (\mathbf{#1})}}
\newcommand*{\trace}[1]{\ensuremath{\mathrm{tr} (\mathbf{#1})}}
\newcommand*{\dev}[1]{(#1 - \bar{#1})}
\newcommand*{\inv}[1]{\ensuremath{\mat{#1}^{-1}}}
\newcommand*{\half}[1]{\ensuremath{\mat{#1}^{1/2}}}
\newcommand*{\invhalf}[1]{\ensuremath{\mat{#1}^{-1/2}}}
\newcommand*{\nvec}[2]{\ensuremath{{#1}_{1}, {#1}_{2},\ldots,{#1}_{#2}}}
\newcommand*{\Beta}{B}
\newcommand*{\Epsilon}{E}
\newcommand*{\period}{\:\: .}
\newcommand*{\comma}{\:\: ,}
\newcommand*{\given}{\ensuremath{\, | \,}}
\newcommand*\widebar[1]{\overline{#1}}

% R stuff
\newcommand{\pkg}[1]{\textsf{#1}}
\newcommand{\Rpackage}[1]{\pkg{#1} package}
\let\proglang=\textsf
\newcommand{\R}{\textsf{R}\xspace}
\end{verbatim}

I like to use \R for data analysis and graphics, but I've also used 
\verb|\proglang{SAS}| which prints as \proglang{SAS}.
In \R my favorite package is \verb|\pkg{ggplot2}| which prints as \pkg{ggplot2}.

I love matrices: 
\verb|\mat{X}, \mat{Y}|
$\rightarrow \mat{X}, \mat{Y}$ are favorites. Transpose, \verb|\mat{X}\trans| $\rightarrow \mat{X}\trans$ turns them on their side. But I can change the notation for transpose $\mathsf{T}$ to \verb|'| with
\begin{verbatim}
	\newcommand{\trans}{\ensuremath{^'}}
\end{verbatim}

There are diagonal ones too, like \verb|\diag{(1,2,3)}| $\rightarrow \diag{(1,2,3)}$ which gives:
$$
\mat{D} = \diag{(1,2,3)} =
\begin{pmatrix}
1 & 0 & 0 \\
0 & 2 & 0 \\
0 & 0 & 3
\end{pmatrix}
$$

Sometimes I like to talk about the \verb|\trace{}| or \verb|rank{}| of a matrix, as in
$$
\trace{D} = 6 \quad\quad
\rank{D} = 3
$$

A data ellipsoid can be defined as:
\begin{verbatim}
\begin{equation}\label{eq:dsq}
	\mathcal{E}_c ( \widebar{\vec{y}},  \mat{S} ) := \{ \vec{y} :
	\dev{\vec{y}}\trans \, \inv{S} \, \dev{\vec{y}} \le c^2 \} \period
\end{equation}
\end{verbatim}

where \verb|\dev{\vec{y}}| and \verb|\inv{S}| simplify the notation. This gives:

\begin{equation}\label{eq:dsq}
	\mathcal{E}_c ( \widebar{\vec{y}},  \mat{S} )
	:= \{ \vec{y} :
	\dev{\vec{y}}\trans \, \inv{S} \, \dev{\vec{y}} \le c^2 \} \period
\end{equation}



\end{document}


