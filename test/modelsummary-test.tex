% Options for packages loaded elsewhere
% Options for packages loaded elsewhere
\PassOptionsToPackage{unicode}{hyperref}
\PassOptionsToPackage{hyphens}{url}
\PassOptionsToPackage{dvipsnames,svgnames,x11names}{xcolor}
%
\documentclass[
  letterpaper,
  DIV=11,
  numbers=noendperiod]{scrartcl}
\usepackage{xcolor}
\usepackage{amsmath,amssymb}
\setcounter{secnumdepth}{-\maxdimen} % remove section numbering
\usepackage{iftex}
\ifPDFTeX
  \usepackage[T1]{fontenc}
  \usepackage[utf8]{inputenc}
  \usepackage{textcomp} % provide euro and other symbols
\else % if luatex or xetex
  \usepackage{unicode-math} % this also loads fontspec
  \defaultfontfeatures{Scale=MatchLowercase}
  \defaultfontfeatures[\rmfamily]{Ligatures=TeX,Scale=1}
\fi
\usepackage{lmodern}
\ifPDFTeX\else
  % xetex/luatex font selection
\fi
% Use upquote if available, for straight quotes in verbatim environments
\IfFileExists{upquote.sty}{\usepackage{upquote}}{}
\IfFileExists{microtype.sty}{% use microtype if available
  \usepackage[]{microtype}
  \UseMicrotypeSet[protrusion]{basicmath} % disable protrusion for tt fonts
}{}
\makeatletter
\@ifundefined{KOMAClassName}{% if non-KOMA class
  \IfFileExists{parskip.sty}{%
    \usepackage{parskip}
  }{% else
    \setlength{\parindent}{0pt}
    \setlength{\parskip}{6pt plus 2pt minus 1pt}}
}{% if KOMA class
  \KOMAoptions{parskip=half}}
\makeatother
% Make \paragraph and \subparagraph free-standing
\makeatletter
\ifx\paragraph\undefined\else
  \let\oldparagraph\paragraph
  \renewcommand{\paragraph}{
    \@ifstar
      \xxxParagraphStar
      \xxxParagraphNoStar
  }
  \newcommand{\xxxParagraphStar}[1]{\oldparagraph*{#1}\mbox{}}
  \newcommand{\xxxParagraphNoStar}[1]{\oldparagraph{#1}\mbox{}}
\fi
\ifx\subparagraph\undefined\else
  \let\oldsubparagraph\subparagraph
  \renewcommand{\subparagraph}{
    \@ifstar
      \xxxSubParagraphStar
      \xxxSubParagraphNoStar
  }
  \newcommand{\xxxSubParagraphStar}[1]{\oldsubparagraph*{#1}\mbox{}}
  \newcommand{\xxxSubParagraphNoStar}[1]{\oldsubparagraph{#1}\mbox{}}
\fi
\makeatother

\usepackage{color}
\usepackage{fancyvrb}
\newcommand{\VerbBar}{|}
\newcommand{\VERB}{\Verb[commandchars=\\\{\}]}
\DefineVerbatimEnvironment{Highlighting}{Verbatim}{commandchars=\\\{\}}
% Add ',fontsize=\small' for more characters per line
\usepackage{framed}
\definecolor{shadecolor}{RGB}{241,243,245}
\newenvironment{Shaded}{\begin{snugshade}}{\end{snugshade}}
\newcommand{\AlertTok}[1]{\textcolor[rgb]{0.68,0.00,0.00}{#1}}
\newcommand{\AnnotationTok}[1]{\textcolor[rgb]{0.37,0.37,0.37}{#1}}
\newcommand{\AttributeTok}[1]{\textcolor[rgb]{0.40,0.45,0.13}{#1}}
\newcommand{\BaseNTok}[1]{\textcolor[rgb]{0.68,0.00,0.00}{#1}}
\newcommand{\BuiltInTok}[1]{\textcolor[rgb]{0.00,0.23,0.31}{#1}}
\newcommand{\CharTok}[1]{\textcolor[rgb]{0.13,0.47,0.30}{#1}}
\newcommand{\CommentTok}[1]{\textcolor[rgb]{0.37,0.37,0.37}{#1}}
\newcommand{\CommentVarTok}[1]{\textcolor[rgb]{0.37,0.37,0.37}{\textit{#1}}}
\newcommand{\ConstantTok}[1]{\textcolor[rgb]{0.56,0.35,0.01}{#1}}
\newcommand{\ControlFlowTok}[1]{\textcolor[rgb]{0.00,0.23,0.31}{\textbf{#1}}}
\newcommand{\DataTypeTok}[1]{\textcolor[rgb]{0.68,0.00,0.00}{#1}}
\newcommand{\DecValTok}[1]{\textcolor[rgb]{0.68,0.00,0.00}{#1}}
\newcommand{\DocumentationTok}[1]{\textcolor[rgb]{0.37,0.37,0.37}{\textit{#1}}}
\newcommand{\ErrorTok}[1]{\textcolor[rgb]{0.68,0.00,0.00}{#1}}
\newcommand{\ExtensionTok}[1]{\textcolor[rgb]{0.00,0.23,0.31}{#1}}
\newcommand{\FloatTok}[1]{\textcolor[rgb]{0.68,0.00,0.00}{#1}}
\newcommand{\FunctionTok}[1]{\textcolor[rgb]{0.28,0.35,0.67}{#1}}
\newcommand{\ImportTok}[1]{\textcolor[rgb]{0.00,0.46,0.62}{#1}}
\newcommand{\InformationTok}[1]{\textcolor[rgb]{0.37,0.37,0.37}{#1}}
\newcommand{\KeywordTok}[1]{\textcolor[rgb]{0.00,0.23,0.31}{\textbf{#1}}}
\newcommand{\NormalTok}[1]{\textcolor[rgb]{0.00,0.23,0.31}{#1}}
\newcommand{\OperatorTok}[1]{\textcolor[rgb]{0.37,0.37,0.37}{#1}}
\newcommand{\OtherTok}[1]{\textcolor[rgb]{0.00,0.23,0.31}{#1}}
\newcommand{\PreprocessorTok}[1]{\textcolor[rgb]{0.68,0.00,0.00}{#1}}
\newcommand{\RegionMarkerTok}[1]{\textcolor[rgb]{0.00,0.23,0.31}{#1}}
\newcommand{\SpecialCharTok}[1]{\textcolor[rgb]{0.37,0.37,0.37}{#1}}
\newcommand{\SpecialStringTok}[1]{\textcolor[rgb]{0.13,0.47,0.30}{#1}}
\newcommand{\StringTok}[1]{\textcolor[rgb]{0.13,0.47,0.30}{#1}}
\newcommand{\VariableTok}[1]{\textcolor[rgb]{0.07,0.07,0.07}{#1}}
\newcommand{\VerbatimStringTok}[1]{\textcolor[rgb]{0.13,0.47,0.30}{#1}}
\newcommand{\WarningTok}[1]{\textcolor[rgb]{0.37,0.37,0.37}{\textit{#1}}}

\usepackage{longtable,booktabs,array}
\usepackage{calc} % for calculating minipage widths
% Correct order of tables after \paragraph or \subparagraph
\usepackage{etoolbox}
\makeatletter
\patchcmd\longtable{\par}{\if@noskipsec\mbox{}\fi\par}{}{}
\makeatother
% Allow footnotes in longtable head/foot
\IfFileExists{footnotehyper.sty}{\usepackage{footnotehyper}}{\usepackage{footnote}}
\makesavenoteenv{longtable}
\usepackage{graphicx}
\makeatletter
\newsavebox\pandoc@box
\newcommand*\pandocbounded[1]{% scales image to fit in text height/width
  \sbox\pandoc@box{#1}%
  \Gscale@div\@tempa{\textheight}{\dimexpr\ht\pandoc@box+\dp\pandoc@box\relax}%
  \Gscale@div\@tempb{\linewidth}{\wd\pandoc@box}%
  \ifdim\@tempb\p@<\@tempa\p@\let\@tempa\@tempb\fi% select the smaller of both
  \ifdim\@tempa\p@<\p@\scalebox{\@tempa}{\usebox\pandoc@box}%
  \else\usebox{\pandoc@box}%
  \fi%
}
% Set default figure placement to htbp
\def\fps@figure{htbp}
\makeatother





\setlength{\emergencystretch}{3em} % prevent overfull lines

\providecommand{\tightlist}{%
  \setlength{\itemsep}{0pt}\setlength{\parskip}{0pt}}



 


\usepackage{booktabs}
\usepackage{longtable}
\usepackage[bf,singlelinecheck=off]{caption}
\usepackage{url}

\usepackage{framed,color}
\definecolor{shadecolor}{RGB}{248,248,248}
% custom colors
\definecolor{darkgreen}{RGB}{1,50,32}


\renewcommand{\textfraction}{0.05}
\renewcommand{\topfraction}{0.8}
\renewcommand{\bottomfraction}{0.8}
\renewcommand{\floatpagefraction}{0.75}

% for tinytable / modelsummary
\usepackage{tabularray}
\usepackage{float}
\usepackage{graphicx}
\usepackage{rotating}
\usepackage[normalem]{ulem}
\UseTblrLibrary{booktabs}
\UseTblrLibrary{siunitx}
\newcommand{\tinytableTabularrayUnderline}[1]{\underline{#1}}
\newcommand{\tinytableTabularrayStrikeout}[1]{\sout{#1}}
\NewTableCommand{\tinytableDefineColor}[3]{\definecolor{#1}{#2}{#3}}


% krantz VF format is ugly!
%\renewenvironment{quote}{\begin{VF}}{\end{VF}}

\let\oldhref\href
\providecommand{\href}[2]{#2\footnote{\url{#1}}}
  
\makeatletter
\newenvironment{kframe}{%
  \medskip{}
  \setlength{\fboxsep}{.8em}
  \def\at@end@of@kframe{}%
  \ifinner\ifhmode%
  \def\at@end@of@kframe{\end{minipage}}%
  \begin{minipage}{\columnwidth}%
  \fi\fi%
  \def\FrameCommand##1{\hskip\@totalleftmargin \hskip-\fboxsep
  \colorbox{shadecolor}{##1}\hskip-\fboxsep
    % There is no \\@totalrightmargin, so:
      \hskip-\linewidth \hskip-\@totalleftmargin \hskip\columnwidth}%
  \MakeFramed {\advance\hsize-\width
    \@totalleftmargin\z@ \linewidth\hsize
    \@setminipage}}%
{\par\unskip\endMakeFramed%
  \at@end@of@kframe}
\makeatother

\renewenvironment{Shaded}{\begin{kframe}}{\end{kframe}}

\usepackage{makeidx}
\makeindex

\usepackage[small,firstabbrev]{authorindex}
\aimaxauthors{5}   % maximum number of authors to index


%%%%%%%%%%%%%%%%%%%%%%%%%%%%%%%%%%%%%%%%%%%%%%%%%%%%%%%%%%%%%%%%%%%%%%%
% Index generation
% Indexentry for a word/phrase (Word inserted into the text)
%%%%%%%%%%%%%%%%%%%%%%%%%%%%%%%%%%%%%%%%%%%%%%%%%%%%%%%%%%%%%%%%%%%%%%%
\newcommand{\IX}[1]{\index{#1}#1}
\newcommand{\ix}[1]{\index{#1}}
\newcommand{\ixmain}[1]{\index{#1|textbf}}
\newcommand{\ixon}[1]{\index{#1|(}}
\newcommand{\ixoff}[1]{\index{#1|)}}

% R functions
\newcommand{\ixfunc}[1]{%
  \index{#1@\texttt{#1()}}%
 }

% R packages:  indexed under both package name and packages!
\newcommand{\ixp}[1]{%
   \index{#1@\textsf{#1} package}%
   \index{package!#1@\textsf{#1}}%
	}

% data sets: 
\newcommand{\ixd}[1]{%
  \index{#1}
  \index{datasets!#1}
  % \index{datasets!@\texttt{#1}}
  % \index{@\texttt{#1} dataset}
  }


\urlstyle{tt}

\usepackage{amsthm}
\makeatletter
\def\thm@space@setup{%
  \thm@preskip=8pt plus 2pt minus 4pt
  \thm@postskip=\thm@preskip
}
\makeatother

\frontmatter
  
  
\usepackage{float}
\usepackage{tabularray}
\usepackage[normalem]{ulem}
\usepackage{graphicx}
\usepackage{rotating}
\UseTblrLibrary{booktabs}
\UseTblrLibrary{siunitx}
\NewTableCommand{\tinytableDefineColor}[3]{\definecolor{#1}{#2}{#3}}
\newcommand{\tinytableTabularrayUnderline}[1]{\underline{#1}}
\newcommand{\tinytableTabularrayStrikeout}[1]{\sout{#1}}
\KOMAoption{captions}{tableheading}
\makeatletter
\@ifpackageloaded{caption}{}{\usepackage{caption}}
\AtBeginDocument{%
\ifdefined\contentsname
  \renewcommand*\contentsname{Table of contents}
\else
  \newcommand\contentsname{Table of contents}
\fi
\ifdefined\listfigurename
  \renewcommand*\listfigurename{List of Figures}
\else
  \newcommand\listfigurename{List of Figures}
\fi
\ifdefined\listtablename
  \renewcommand*\listtablename{List of Tables}
\else
  \newcommand\listtablename{List of Tables}
\fi
\ifdefined\figurename
  \renewcommand*\figurename{Figure}
\else
  \newcommand\figurename{Figure}
\fi
\ifdefined\tablename
  \renewcommand*\tablename{Table}
\else
  \newcommand\tablename{Table}
\fi
}
\@ifpackageloaded{float}{}{\usepackage{float}}
\floatstyle{ruled}
\@ifundefined{c@chapter}{\newfloat{codelisting}{h}{lop}}{\newfloat{codelisting}{h}{lop}[chapter]}
\floatname{codelisting}{Listing}
\newcommand*\listoflistings{\listof{codelisting}{List of Listings}}
\makeatother
\makeatletter
\makeatother
\makeatletter
\@ifpackageloaded{caption}{}{\usepackage{caption}}
\@ifpackageloaded{subcaption}{}{\usepackage{subcaption}}
\makeatother
\usepackage{bookmark}
\IfFileExists{xurl.sty}{\usepackage{xurl}}{} % add URL line breaks if available
\urlstyle{same}
\hypersetup{
  pdftitle={Test modelsummary},
  colorlinks=true,
  linkcolor={blue},
  filecolor={Maroon},
  citecolor={Blue},
  urlcolor={Blue},
  pdfcreator={LaTeX via pandoc}}


\title{Test modelsummary}
\author{}
\date{}
\begin{document}
\maketitle


This example gave errors when compiled to PDF

\begin{Shaded}
\begin{Highlighting}[]
\FunctionTok{library}\NormalTok{(modelsummary)}
\FunctionTok{data}\NormalTok{(Prestige, }\AttributeTok{package=}\StringTok{"carData"}\NormalTok{)}
\CommentTok{\# Reorder levels of type}
\NormalTok{Prestige}\SpecialCharTok{$}\NormalTok{type }\OtherTok{\textless{}{-}} \FunctionTok{factor}\NormalTok{(Prestige}\SpecialCharTok{$}\NormalTok{type, }
                        \AttributeTok{levels=}\FunctionTok{c}\NormalTok{(}\StringTok{"bc"}\NormalTok{, }\StringTok{"wc"}\NormalTok{, }\StringTok{"prof"}\NormalTok{)) }
\CommentTok{\#str(Prestige)}
\end{Highlighting}
\end{Shaded}

For illustration, I'll consider three models for the \texttt{Prestige}
data of increasing complexity:

\begin{itemize}
\tightlist
\item
  \texttt{mod1} fits the main effects of the three quantitative
  predictors;
\item
  \texttt{mod2} adds the categorical variable \texttt{type} of
  occupation;
\item
  \texttt{mod3} allows an interaction of \texttt{income} with
  \texttt{type}.
\end{itemize}

\begin{Shaded}
\begin{Highlighting}[]
\NormalTok{mod1 }\OtherTok{\textless{}{-}} \FunctionTok{lm}\NormalTok{(prestige }\SpecialCharTok{\textasciitilde{}}\NormalTok{ education }\SpecialCharTok{+}\NormalTok{ income }\SpecialCharTok{+}\NormalTok{ women,}
           \AttributeTok{data=}\NormalTok{Prestige)}
\NormalTok{mod2 }\OtherTok{\textless{}{-}} \FunctionTok{lm}\NormalTok{(prestige }\SpecialCharTok{\textasciitilde{}}\NormalTok{ education }\SpecialCharTok{+}\NormalTok{ women }\SpecialCharTok{+}\NormalTok{ income }\SpecialCharTok{+}\NormalTok{ type,}
           \AttributeTok{data=}\NormalTok{Prestige)}
\NormalTok{mod3 }\OtherTok{\textless{}{-}} \FunctionTok{lm}\NormalTok{(prestige }\SpecialCharTok{\textasciitilde{}}\NormalTok{ education }\SpecialCharTok{+}\NormalTok{ women }\SpecialCharTok{+}\NormalTok{ income }\SpecialCharTok{*}\NormalTok{ type,}
           \AttributeTok{data=}\NormalTok{Prestige)}
\end{Highlighting}
\end{Shaded}

The main function \texttt{modelsummary()} can produce highly
customizable tables of coefficients in a wide variety of output formats,
including HTML, PDF, LaTeX, Markdown, and MS Word. You can select the
statistics displayed for any model term with the \texttt{estimate} and
\texttt{statistic} arguments.

\begin{Shaded}
\begin{Highlighting}[]
\FunctionTok{modelsummary}\NormalTok{(}\FunctionTok{list}\NormalTok{(}\StringTok{"Model1"} \OtherTok{=}\NormalTok{ mod1),}
  \AttributeTok{coef\_omit =} \StringTok{"Intercept"}\NormalTok{,}
  \AttributeTok{shape =}\NormalTok{ term }\SpecialCharTok{\textasciitilde{}}\NormalTok{ statistic,}
  \AttributeTok{estimate =} \StringTok{"\{estimate\} [\{conf.low\}, \{conf.high\}]"}\NormalTok{,}
  \AttributeTok{statistic =} \FunctionTok{c}\NormalTok{(}\StringTok{"std.error"}\NormalTok{, }\StringTok{"p.value"}\NormalTok{),}
  \AttributeTok{fmt =} \FunctionTok{fmt\_statistic}\NormalTok{(}\StringTok{"estimate"} \OtherTok{=} \DecValTok{3}\NormalTok{, }\StringTok{"conf.low"} \OtherTok{=} \DecValTok{4}\NormalTok{, }\StringTok{"conf.high"} \OtherTok{=} \DecValTok{4}\NormalTok{),}
  \AttributeTok{gof\_omit =} \StringTok{"."}\NormalTok{)}
\end{Highlighting}
\end{Shaded}

\begin{table}

\caption{\label{tbl-modelsummary1}Table of coefficients for the main
effects model.}

\centering{

\centering
\begin{tblr}[         %% tabularray outer open
]                     %% tabularray outer close
{                     %% tabularray inner open
colspec={Q[]Q[]Q[]Q[]},
cell{2}{2}={}{halign=c,},
cell{2}{3}={}{halign=c,},
cell{2}{4}={}{halign=c,},
cell{3}{2}={}{halign=c,},
cell{3}{3}={}{halign=c,},
cell{3}{4}={}{halign=c,},
cell{4}{2}={}{halign=c,},
cell{4}{3}={}{halign=c,},
cell{4}{4}={}{halign=c,},
cell{5}{2}={}{halign=c,},
cell{5}{3}={}{halign=c,},
cell{5}{4}={}{halign=c,},
cell{1}{3}={}{halign=c, halign=c,},
cell{1}{4}={}{halign=c, halign=c,},
cell{2}{1}={}{halign=l,},
cell{3}{1}={}{halign=l,},
cell{4}{1}={}{halign=l,},
cell{5}{1}={}{halign=l,},
cell{1}{1}={}{halign=l, halign=c,},
cell{1}{2}={c=3,}{halign=c, halign=c, halign=c,},
}                     %% tabularray inner close
\toprule
& Model1 &  &  \\ \cmidrule[lr]{2-4}
& Est. & S.E. & p \\ \midrule %% TinyTableHeader
education & \num{\num{4.187}} [\num{\num{3.4153}}, \num{\num{4.9580}}] & \num{\num{0.389}} & \num{\num{0.000}} \\
income & \num{\num{0.001}} [\num{\num{0.0008}}, \num{\num{0.0019}}] & \num{\num{0.000}} & \num{\num{0.000}} \\
women & \num{\num{-0.009}} [\num{\num{-0.0692}}, \num{\num{0.0514}}] & \num{\num{0.030}} & \num{\num{0.770}} \\
\bottomrule
\end{tblr}

}

\end{table}%




\end{document}
